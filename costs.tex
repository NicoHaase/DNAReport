\section{Performance influence}
	The reworked class structure leads to a more readable and intuitive way to use DNA for
	the programmer, but at which cost? The data structures are constructed using reflection,
	which causes some
	slowdown\footnote{\url{http://stackoverflow.com/questions/435553/java-reflection-performance}
	reports about a slowdown in the magnitude of factor 10}, and having the profiler
	enabled also causes additional work. 
	
	\ \\ 	
	\centering
	\begin{tabular}{lll}
		Task & using usual inialization & using reflection \\
	    500 000 \texttt{ArrayList}s, putting one string into them & 
	    	$\sim 125$ msec & $\sim 650$ msec   \\
	    500 000 \texttt{HashSet}s, putting one string into them  &
	    	$\sim 310$ msec & $\sim 800$ msec \\
		~ & ~ & \\
		Task & AspectJ disabled & AspectJ enabled \\
    	Metric \texttt{Undirected\allowbreak Clustering\allowbreak
			CoefficientU} & $\sim 4100$ msec & $\sim 4800$ msec ($+ 17\%$) \\
		Metric \texttt{Degree\allowbreak DistributionR} &
			$\sim 3300$ msec & $\sim 3900$ msec ($+ 18\%$) \\
		Metric \texttt{Undirected\allowbreak Shortest\allowbreak PathsR} &
			$\sim 7200$ msec & $\sim 8800$ msec ($+ 22\%$) \\
    \end{tabular}
    
	\ \\
	Runtime comparisons for the profiler part strongly rely on the monitored use
	case, common parts: see section \ref{sec:testSetup}
