\section{Performance influence}
	The reworked class structure leads to a more readable and intuitive way to use DNA for
	the programmer, but at which cost? The data structures are constructed using reflection,
	which causes some
	slowdown\footnote{\url{http://stackoverflow.com/questions/435553/java-reflection-performance}
	reports about a slowdown in the magnitude of factor 10}, and having the profiler
	enabled also causes additional work.
	
	\begin{itemize}
		\item Initialization of 500 000 \texttt{ArrayList}s and putting one string into them:
			$\sim 125$ msec, compared to $\sim 650$ msec using reflection
		\item Running the same analysis using \texttt{HashSet}: $\sim 310$ msec compared to
			$\sim 800$ msec
		\item Runtime comparisons for the profiler part strongly rely on the monitored use case,
			common parts: see section \ref{sec:testSetup}
		\item Running the metric \texttt{Undirected\allowbreak Clustering\allowbreak
			CoefficientU} takes $\sim 4100$ msec without profiling and $\sim 4800$ msec with
			profiling enabled ($+ 17\%$)
		\item Running the metric \texttt{Degree\allowbreak DistributionR} takes $\sim 3300$
			msec without profiling and $\sim 3900$ msec with profiling enabled ($+ 18\%$)
		\item Running the metric \texttt{Undirected\allowbreak Shortest\allowbreak PathsR} takes
		$\sim 7200$ msec without profiling and $\sim 8800$ msec with profiling enabled ($+ 22\%$)
	\end{itemize}