\section{Motivation}

Former DNA version had data structures hard coded into Java classes. Classes like
\texttt{DirectedNode} were extended to store internal structures in specific data
structures, such that \texttt{DirectedNodeAl} stored its edges in an arraylist. For each
internal data structure, a dedicated class needs to be written. Classes containing the
whole graph were named after their internal data structures, eg.
\texttt{DirectedGraphAlAl}.

\begin{itemize}
	\item Adding a data structure is much work, as in addition to the data structure
		itself many classes for its usage need to be created
	\item Combining data structures leads to chaos, as for each combination of
		data structures, a specific graph class needs to be created
\end{itemize}


