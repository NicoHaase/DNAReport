\section{Building DNA}
	\begin{itemize}
		\item For building DNA in Eclipse, one should install the "AspectJ Development Tools"
			over the Eclipse marketplace. The DNA project in Eclipse needs to be converted to an
			AspectJ project (right click on the project, select Configure $\rightarrow$ Convert to
			AspectJ project). Now the profiler can be told to count accesses through calling
			\texttt{Profiler.\allowbreak activate();}
		\item Remark: while the project is configured with AspectJ, the additional code is
			always woven in. Not activating the profiler saves some runtime (as computing the
			recommendations is the most time consuming part of the profiler), but there is a
			runtime difference between having AspectJ completely disabled and having it enabled,
			but not activating the profiler
		\item Building the project without Eclipse requires the AspectJ compiler available at
			\url{http://eclipse.org/aspectj/} and some modifications for the proper locations of
			AspectJ and JUnit in \texttt{build.xml}. Afterwards, on can compile the code and create
			JARs using the specified build targets. See \ref{sec:antBuildFile} for an example file
	\end{itemize}
