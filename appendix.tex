\section{Appendix}
\subsection{AspectJ} \label{sec:aspectj}
	\begin{itemize}
		\item AspectJ can be used to add functionalities using completely
			seperate files that are woven into existing code on compile time
		\item Defined through point cuts that hook into the existing code and aspects that
			extend the code
	
			\begin{verbatim}
				pointcut methodCall() :
				  execution(* Class.Method())
				  
				before() : methodCall() {
				  System.out.println("Calling Method");
				}
			\end{verbatim}
		
		\item On compile time, the aspects are woven into the byte code transparently
		
	\end{itemize}
\subsection{Test setup} \label{sec:testSetup}
	\begin{itemize}
		\item Common parts for the analysis: random undirected weighted graph with 1 000 nodes,
			10 000 edges
		\item Common batch generation, if not stated otherwise: random batch generation with
			1 000 node additions, 500 node removals, 50 changes of node weights, 2 000 edge
			additions, 500 edge removals, 50 changes of edge weights
		\item Runtimes are averaged over five series generations, running 3 runs with 4 batches
			each
		\item Sources are available via \url{https://github.com/NicoHaase/}
		\item Test setup: Windows 7 on a Dell laptop, Intel i5 dual core CPU with 2.7 Ghz, Java
			1.7.0\_45
	\end{itemize}
	
\subsection{Example output} \label{sec:exampleOutput}
	\verbatiminput{___metric.profiler.values}