\section{Evaluation / demonstration}
	The profiler should give one the ability to run DNA with an optimized combination of data
	structures. In this section, we want to demonstrate the capabilites of the recommender.
	After having already evaluated some very simple cases in section \ref{sec:performance},
	we additionally chose two more scenarios for this to evaluate the recommender. We
	initially select data structures for the global node and edge list and the node-local
	edge list as bad as possible to demonstrate the later enhancements.
	
	\todoInline{Re run with new size}
	
	In the first scenario, the graph should grow from batch to batch by 1000 nodes and 2000
	edges without removing anything. Before each graph update, the degree distribution is
	refreshed. We store all lists in arrays as it is expensive to add entries to an array:
	the first step is checking whether the element is already contained in that array by
	iterating over the whole array (linear complexity - other data structures can do this
	with static complexity), the second step is adding the element. The calculated complexity
	for one example run with this data structure combination is $549534*O(1) + 324576*O(d) +
	295066*O(E)$, the runtime can be averaged to $\sim 9000$ msec. Using arrays to store the
	global node list and HashSets for both edge lists, the complexity can be estimated to
	$1006888*O(1)$. Using the recommended combination leads to a complexity of $844477*O(1)$
	and an averaged runtime of $\sim 2000$ msec ($- 77\%$).
	
	\todoInline{check data structures in Metric.java}
	
	In the second scenario, it should shrink from a large graph, computing shortest paths
	alongside. \todoInline{Describe the second scenario}
	
	
