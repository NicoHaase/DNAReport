\section{Data structures}
	\begin{itemize}
		\item Data structure classes to hold single properties (nodes and edges) were decoupled
			from the graph and node classes by using a common interface for the ``storage''
			classes. At each possible place, this interface is used instead of concrete classes
		\item The common interface holds methods for all actions that are necessary for common
			accesses, like \texttt{add}, \texttt{contains}, \texttt{remove}, \texttt{size},\ldots,
			and internally uses another data structures (eg. \texttt{DArray} stores everything in
			an array, \texttt{DHashSet} in a hashset,\ldots)
		\item Extended interfaces hold methods that are not available on all data structures,
			like \texttt{getRandom}. A data structure that makes use of this constraint in the base
			interface is a Bloom filter which probabilistically stores only flags for contained
			elements, but not a pointer to the elements itself, for less memory consumption
		\item A class holds the currently used storage classes for the global edge and node
			lists, and the node-local edge lists.
		\item Generating a new instance of a storage class is done through
			\texttt{GraphDataStructure}, using Reflection such that the used storage class can be
			exchanged on runtime
		\item Currently included data structures: pure array, \texttt{ArrayList},
			\texttt{HashMap}, \texttt{HashSet}, \texttt{LinkedList}
	\end{itemize}
	
	This concept allows us to add more data structure classes and combine and exchange them
	very easily, without having a cluttered namespace
